\section{Implementation Plan}
% The implementation plan is a preliminary breakdown of the work that is to be
% done in the remainder of the project. You should identify a set of milestones
% and provide a realistic estimate of when each of these should be completed if
% all goes well. It should also detail fallback positions in case any stage of the
% development goes wrong. You may feel, in the early stages of your project work,
% that the times in this plan are guesses. However you will find as the project
% progresses that keeping track of and revising your initial estimates, and if
% necessary altering the proposed work, is a vital way to ensure that the project
% is finished in time. In projects with heavy implementation content you should
% document what you have already completed.


\subsection{Milestones}
The initial deliverable for the engineering side involves running a simple
program on the FPGA through the HPS with the FPGA frequency being controllable.
After this, the next critical step would be making sure the modules under test
will be the point of failure and not the testbench.
This would include some research on ways in improving the speed of feeding
inputs to the arithmetic units, and checking its outputs.
Once this could be confirmed, we can start adding a selection of different functionalities.

\begin{enumerate}
  \item Running standard benchmarks;
  \item Running key algorithms or their components;
  \item Experiment with other power efficiency improving techniques,
        such as undervolting;
  \item Add support for configurable radix arithmetic;
  \item Allow graceful failures for the testbench in case of unintended
        behaviour for the arithmetic modules;
  \item Add an interactive UI to control the voltage and frequency at run time
        and examine the DUT’s behaviour;
\end{enumerate}

Depending on the time situation, more or less items on this list may be fulfilled.
The method of evaluation will be discussed later in the evaluation chapter.

\subsection{Work to Date}