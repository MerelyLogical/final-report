\section{Implementation Plan}
% The implementation plan is a preliminary breakdown of the work that is to be
% done in the remainder of the project. You should identify a set of milestones
% and provide a realistic estimate of when each of these should be completed if
% all goes well. It should also detail fallback positions in case any stage of the
% development goes wrong. You may feel, in the early stages of your project work,
% that the times in this plan are guesses. However you will find as the project
% progresses that keeping track of and revising your initial estimates, and if
% necessary altering the proposed work, is a vital way to ensure that the project
% is finished in time. In projects with heavy implementation content you should
% document what you have already completed.


\subsection{Milestones}
The project will be divided into three main phases.
The initial one is of research and learning.
This is to lay the foundation to the project, and is represented by tasks 2 and
3 in the timeline.
The second phase is building the core product of the project.
This includes tasks 4, 5 and 6 in the timeline.
The final phase is enhancing the product, represented by tasks 7, 8, and 9.

\subsection{Timeline}
\begin{figure*}[!b]
  \centering
  \begin{ganttchart}[
  vgrid={*{6}{draw=none}, dotted},
  x unit=.05cm,
  y unit title=.6cm,
  y unit chart=.6cm,
  title height=.75,
  title top shift=0,
  today=2019-01-28,
  time slot format=isodate,
  ]{2018-10-01}{2019-06-30}
  \gantttitlecalendar{year, month=name} \\
  % REVISIT: add formatting to timeline
  \ganttbar{Term Time}          {2018-10-01}{2018-12-16}   % wk40-50
  \ganttbar{}                   {2019-01-07}{2019-03-24}   % wk02-12
  \ganttbar{}                   {2019-04-29}{2019-06-30} \\% wk18-26

  \ganttbar{Examinations}       {2019-01-05}{2019-01-09}   % wk01
  \ganttbar{}                   {2019-03-11}{2019-03-24}   % wk11-12
  \ganttbar{}                   {2019-04-22}{2019-05-02}   % wk17-18
  \ganttbar{}                   {2019-05-13}{2019-05-20} \\% wk20-21
  % REVISIT: Adjust timeline for easter holiday
  \ganttbar{Background Research}{2018-10-29}{2019-01-20} \\% wk44-03
  \ganttbar{Learning the Tools} {2018-11-03}{2018-12-07} \\% wk45-49
  \ganttbar{Testbench Structure}{2019-01-07}{2019-01-27} \\% wk02-04
  \ganttbar{Variable Frequency} {2019-01-28}{2019-02-17} \\% wk05-07
  \ganttbar{Benchmarks}         {2019-02-18}{2019-03-10} \\% wk08-10
  \ganttbar{Configurable Radix} {2019-03-25}{2019-04-14} \\% wk13-15
  \ganttbar{Handling Failures}  {2019-04-15}{2019-04-21}   % wk16
  \ganttbar{}                   {2019-05-03}{2019-05-12} \\% wk18-19
  \ganttbar{Interactive UI}     {2019-05-21}{2019-06-09} \\% wk21-23

  \ganttbar{Interim Report}     {2019-01-05}{2019-01-28} \\% wk01-09
  \ganttbar{Draft Report}       {2019-04-01}{2019-06-03} \\% wk14-23
  \ganttbar{Final Report}       {2019-04-01}{2019-06-19} \\% wk14-25
  \ganttbar{Presentation}       {2019-05-24}{2019-06-26}   % wk21-26
\end{ganttchart}
  \caption{Project Timeline}
  \label{Timeline}
\end{figure*}

In order to track the progress and success of the project, the difficulties
of the deliverables need to be analysed first.
Figure \ref{Timeline} provides a visualisation of the project timeline.
Some slack is included into the plan to reduce risk and give potential for
extension.

% REVISIT: coloured tasks!

\subsubsection{\textbf{Term Time and Examinations}}
Time available for the project varies greatly throughout the year.
The greatest factors will be term time and examinations.
The time needed for revision has been marked in the chart.
These times will allow minimal progress of the project.

During term time, there will be coursework deadlines, which will also
negatively affect the time that can be allocated to this project.
One coursework module was selected for Autumn and three were picked for Summer.
As such, it is expected that the progress will be somewhat slower during Autumn
but significantly slower during Spring.
To make up for the time lost, a part of Christmas was used and a few weeks
from Easter will also be committed towards the project.

The list of tasks were then laid out onto the timeline.
This was done according to their expected difficulty and the expected
availability to work on those tasks.

\subsubsection{\textbf{Background Research}}
To fill in the background knowledge required to work on this project, the first
months were spent on reading textbooks and papers.
The research was to provide context and motivation for the project.
It also provided an overview of the field of research and some understanding
towards the current state-of-the-art.
Most importantly, it offered the necessary knowledge needed for this project.
The result of this research was summarised in chapter 3.

\subsubsection{\textbf{Learning the Tools}}
Due to the lack of experience in programming hybrid SoCs and the lack of
knowledge in the current state-of-the-art digital arithmetic designs,
significant effort was spent on researching and learning the skills
necessary to carry out the project.
This involved building a small testing system on the board.
Details regarding this testing system can be found in section 5.3.

\subsubsection{\textbf{Testbench Structure}}
Once comfortable with the tools, the main design of the testbench can start.
This task will be the foundation of this project, as it will provide a basis
for all of the following features.
A skeletal testbench should be complete and functional at the end of this
task.
This means an operator module matching the correct specifications can
be loaded into the testbench to run basic tests.
To reduce risk and uncertainty at this stage in the project, this basic
testbench is not required to have the full flexibility of the final product
in terms of compatibility.
While it is still undesirable to hardcode parameters and features,
the support for DUTs different from the one built by the other project will
only be addressed later as task 7.

\subsubsection{\textbf{Variable Frequency}}
As the project seeks to quantify performance across a range of frequencies,
the most important feature of the system will be the ability to vary this
parameter.
This will be done by controlling the PLLs with the HPS~\cite{Altera4}.
Once implemented, tests will be run to explore and confirm the maximum
frequency for which the testbench can remain reliable.
If it does not meet the planned target, the testbench may need to be redesigned,
and the project would be under some risk.

While most of the later sections of the project can be selectively added or
removed from the scope relatively easily, this initial setup of the testbench
structure will always remain critical to any further improvements.
It is thus vital that the bare minimum system is completed early.
To ensure that this happens, this task and the testbench structure will be
placed in the highest priority before its completion, and any blocking issue
will be discussed with the supervisor if it cannot be resolved after reasonable
effort.

\subsubsection{\textbf{Benchmarks}}
Once the testbench can reach a satisfying performance, more intricate tests and
benchmarks can be designed.
These aims to reflect the system's performance running meaningful compute
tasks.
The systems enveloping the arithmetic modules could be stressed with popular
algorithms to evaluate real-life obtainable speedups.
In addition to running better tests, we also aim to obtain better data from
the tests.
The minimum result required here would be numerical information on power
consumption, FPGA resources required, and the data throughput.
% REVISIT JD: how to obtain power consumption number?

The last three tasks form the core of this project.
In other words, all three tasks must be completed for a functional product.
The following tasks will be mostly considered as useful extensions.
While not as vital as the core tasks, completion of the following tasks would
greatly improve usability, and thus are still relevant to the success
of the final product.

\subsubsection{\textbf{Configurable Modules}}
So far, only modules of a specific I/O width and numerical representation can
be tested.
It might be interesting to explore arithmetic modules with other configurations.
To allow the testbench to be used for further experiments or future projects,
it will be helpful to have a configurable testbench.
Qsys components can be configured with a Hardware Component Definition
File~\cite{Altera5}.
The plan is to build the testbench as a Qsys component, then use Qsys Component
Editor as an interface for configuration.
This is because the Component Editor provides a friendly interface for
configuration directly from analysing the testbench design file.

\subsubsection{\textbf{Handling Failures}}
Another improvement to the testbench is related to how it handles failures
in the module.
It would be much more insightful for the user if a more insightful failure
message were provided in addition to just a simple failure rate.
This could include examples of failed output against expected output,
or statistical data describing the pattern of failures.
This additional logic required may degrade performance of the testbench,
so it would be useful for the verbosity of this information to be configurable
by the user.

\subsubsection{\textbf{Interactive UI}}
If time allows for even greater user experience enhancements, real-time
interactive graphical user interface could be constructed for the final
demonstration.
This would visualise the reduction of the module's accuracy as the user
increases the clock rate.
However, this would take significant time and effort, and this task will
be re-evaluated when the project progresses to the stage.
A time-saving yet functional alternative would be a command-line interface
with a well-documented user guide.

\subsubsection{\textbf{Reports and Presentation}}
The reports and the presentation are the most visible in all deliverables of
this entire process.
As such, while not directly contributing to the progress of the project,
they are still vital to its success.
The reports will be written alongside the engineering process.
At the end, around a week of time will be spent solely on completing and
polishing up the final report.
This should allow ample time for a well-organised submission.

The week after the final report will be used for the presentation.
This involves preparing a slide deck, a demonstration, and a script.
During the demonstration, the product will be exhibited in front of the
audience.
The details of the exhibition largely depends on the status of task 9.

\subsection{Work to Date}
% REVISIT: include the use of git somewhere

Before any engineering work was done towards the final product, a small module
was built to learn the environment.
This module needed to be simple yet covered enough grounds to provide as much
learning during the process as possible without taking up too much actual
development time towards the product.
As the greatest unfamiliarity was with the interaction across the HPS-FPGA
bridge, a simple hardware accelerated adder was made for this training.

\subsubsection{\textbf{FPGA Side}}
Programming the FPGA to communicate with the HPS is no trivial task.
Luckily, there exists a golden system reference design~\cite{Rocket1} for
the board in use for this project.
Unfortunately, support for certain versions of Quartus are missing from
the GSRD download database, including the one used for this project, 16.0.
While the design can be opened with a different version of the software,
it causes a series of conflicts usually related to using IP cores that
have changed over the iterations.
To circumvent this issue cleanly, GSRD version 14.1 was downloaded and compiled
on a separate install of Quartus II 14.1.
This allowed the reference design to be studied in detail, and the sections
required for this project to be rebuilt with Quartus Prime 16.0.

From the perspective of the FPGA, The HPS exposes three bridges for
connections~\cite{Altera6}.
As this is a relatively simple task, the lightweight bridge was used.
Module \texttt{altera\_hps} exposes the master of this bridge as
\texttt{h2f\_lw\_axi\_master}.
Next, the actual hardware adder was built and integrated as a hardware
module in Qsys with a matching interface.
A simple adder can produce its result after one clock cycle.
This greatly simplifies the logic required for the Avalon slave interface.
The logic for the control and data signals were thus written according to the
interface specifications~\cite{Intel3}.
Following the naming conventions for the signals allows Qsys Component Editor
to automatically detect the Avalon slave from this module at analysis.
This saves the troubles of editing the \texttt{\_hw.tcl} file.
To experiment with module configuration, the adder was designed with variable
width.

The adder was then instantiated and connected to the rest of the system with
two clicks in Qsys.
From there, Qsys could generate the HDL for the entire system, which was then
compiled to a bitstream file.
With the bitstream ready, the work now shifted to the HPS.

\subsubsection{\textbf{HPS Side}}
The HPS runs Ubuntu and a Bash script has been written to load the bitstream
onto the FPGA.
Next, a program was written in Python to test the hardware design from the HPS.
The interfaces are mapped onto the physical memory, thus they can be accessed
by opening \texttt{/dev/mem}.
Checking against the specifications~\cite{Altera6}, the lightweight master is
at \texttt{0xFF20\_0000}.
Qsys allocates the memory spaces of modules relatively, so when it reports
that the adder has been placed at \texttt{0x0010\_0000}, it is physically at
\texttt{0xFF30\_0000}.
The adder was designed to have its two inputs at \texttt{0x00} and
\texttt{0x10} and its output at \texttt{0x20}, which were assigned by Qsys
relatively to \texttt{0xFF30\_0000}.

With the memory mapping understood, the program was designed to closely
mirror this relative relationship between the modules using classes.
For example, the adder defines its output at \texttt{0x20}, but its read and
write methods are inherited from an AXI class that brings it to the
correct physical address by adding the address of the bridge defined in it.
This parallel between software and hardware should be helpful as the product
gets more complex.

To verify what I have built and learnt was correct, 1000 add operations were
executed separately with and without the hardware acceleration of the FPGA.
The results were compared and confirmed that this training module worked.
While called hardware acceleration, the FPGA actually had worse performance
than the HPS in this testing case.
The CPU is reasonably efficient in calculating additions, while calculating
on the FPGA requires the adder I/O data going through the HPS-FPGA bridge.
This incurs a significant overhead, thus slowing it down.

% REVISIT: add test data here
