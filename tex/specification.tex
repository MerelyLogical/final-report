\section{Project Specification}

\subsection{Project Organisation}
This project is a part of a larger project investigating the effect of using high-radix number representation with online arithmetic operators.
The overarching aim involves implementing such a system on an FPGA and quantifying its performance improvements.
This is achieved through two individual projects, vertically split from the enveloping project.
One shall design the arithmetic operator modules, while the other shall design a system from the top-level to test and evaluate these operators.
This project deals with the system-level issues.

As this project progresses in parallel with the designing of the operator modules, it is necessary to decouple the two projects so that, being individual projects, they can be evaluated individually.
The success of one project should not be restricted by the status of the other.
To this end, the goal of the system-level design is more focussed on its functionalities and robustness.
This relationship and its effect on the evaluation will be examined further in the evaluation chapter of this report.

To ensure the two products will work together once they are both complete, a common interface is agreed upon.
The interface will be done using Qsys.
The unit-level project will build different operators, which can have varying arithmetic functions and designs.
These can be packaged into individual Qsys modules, as adders, multipliers, or dividers.
Alternatively they can also be delivered as a single module taking two operands and an instruction that is one of the four basic arithmetic operations.
These will then become the DUTs of the testbench.

\subsection{Deliverables}
At the end of the project, the system should be able to perform the following:
\begin{enumerate}
  \item Connect to the arithmetic modules as its input;
  \item Generate and run tests on these modules;
  \item Vary the frequency of the FPGA;
  \item Evaluate its performance.
\end{enumerate}
