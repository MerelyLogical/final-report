\chapter{Software Design}

\section{Interface Design}
The software should provide a layer of abstraction so that the user can run tests without worrying about too many technical details.
The abstraction needs to be intuitive, but it should not compromise on the level of control given to the user.
Based on the hardware design, a list of items that should be controllable by the user at the software level is drafted.

\begin{enumerate}
  \setlength\itemsep{0pt}
  \item Operating mode (manual or auto)
  \item Manual inputs in manual mode
  \item Bit set/clear control in auto mode
  \item Test duration
  \item PLL frequency
\end{enumerate}

All other variables such as the exact memory locations accessed and the binary values being read and written should be hidden away by the interface.
For example, while the \texttt{freeze} signal in the scoreboard is exposed to the HPS-FPGA Bridge, the user will not have to manually assert the signal.
The signal should be automatically asserted by the software before results were collected for the scoreboard, ensuring the scoreboard registers are not still counting during the read process.
Allowing manual control of this signal would only require unnecessary effort from the user, thus it should be one of the items abstracted away by the software.
However, the code still needs to be constructed in a way, such that an expert user attempting to modify the bridge interface should be able to do so without much trouble.

Two options were considered here.

\subsection{Configuration File}
One possible arrangement is to set up a configuration file for the software.
We can have a YAML file that lists all the default values for the controls.
These values can then be edited by the user to their liking.
YAML is chosen as its Python-like appearance and its easiness to read and edit.
Python, the language which the software is written in, also has good support for parsing YAML files with the \texttt{pyyaml} library.

\begin{figure}[H]
  \centering
  \begin{minipage}[b]{.35\linewidth}
  \begin{verbatim}
mode: auto
bitset:
  a: 00000001
  b: 00000000
bitclr:
  a: 00000000
  b: 00000001
input:
  - a: 00000000
    b: 00000000
freq: 200
runtime: 60
  \end{verbatim}
  \caption{Config File 1}
  \label{autoYMAL}
  \end{minipage}%
  \begin{minipage}[b]{.35\linewidth}
  \begin{verbatim}
mode: manual
bitset:
  a: 00000000
  b: 00000000
bitclr:
  a: 00000000
  b: 00000000
input:
  - a: deadf00d
    b: fadef00d
  - a: feedf00d
    b: cafef00d
freq: 100
runtime: 5
  \end{verbatim}
  \caption{Config File 2}
  \label{manYMAL}
  \end{minipage}
\end{figure}

Looking at the configuration file excerpts, Figure \ref{autoYMAL} will make the test run in auto mode for 60 seconds at 200MHz.
Input \texttt{a} will be always odd, and input \texttt{b} will always be even.
Figure \ref{manYMAL} on the other hand will make the test run in manual mode, in which the testbench will go through the list of inputs stated under the input key.
Once the list is exhausted, the test will terminate.

This method is great for setting up one or two tests, but to scale this up to series of tests, the user may have to generate a collection of such configuration files.
The test software can then scan a folder instead of just a single file and run all tests.

If there happened to be significant extensions to the interface in the future, the user will also have to go through many configuration options that maybe irrelevant to the test case.
This increases the cognitive load unnecessarily, reducing the user-friendliness of the design.

\subsection{Read-eval-print Loop}

Alternatively, the software can be structured into a read-eval-print loop (REPL).
This is a simple interactive command line interface, where the program reads a single user expression, evaluates it, and prints out a response.
The software then waits for the next user input, and so the loop continues until the user decides to stop.

\begin{figure}[H]
  \centering
  \begin{minipage}[b]{.5\linewidth}
  \begin{verbatim}
> mode auto
> bitset a 00000001
> bitclr b 00000001
> freq 200
PLL Configured to 200.00MHz
> run 60
Results: ...
  \end{verbatim}
  \caption{CLI Excerpt 1}
  \label{autoREPL}
  \end{minipage}%
  \begin{minipage}[b]{.5\linewidth}
  \begin{verbatim}
> mode manual
> input a 00010001
> input b 000a000c
> input a feedf00d
> input b cafef00d
> freq 100
PLL Configured to 100.00MHz
> run 1
Results: ...
  \end{verbatim}
  \caption{CLI Excerpt 2}
  \label{manREPL}
  \end{minipage}
\end{figure}

This allows the user to intuitively command the software.
Figure \ref{autoREPL} and Figure \ref{manREPL} shows possible commands the user may use to achieve the same effect as the previous option with configuration files.
The user can now go directly to the option that needs to be changed, and if a mistake with the configuration was found when looking at the test results, the user can immediately make adjustments within the software's command line interface (CLI).
This means the user can be more exploratory, as REPL is more interactive and direct than having to reset, edit a file, run again, and then debug.

In order to scale this option, the software can be easily modified to also accept a macro file as well.
Instead of typing in individual commands in the CLI, the user can enter all the necessary commands as lines of a file, and feed that into the software.
The macro file can then be scaled and automated so series of tests can be easily ran with a single instruction from the user.

Being the more user-friendly and more scalable option, this option will be implemented in the next phase of the project.
