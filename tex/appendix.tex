\chapter{User Guide}
\label{appendix:guide}

\section{Introduction}

This project provides an extensible framework for at-speed evaluation of arithmetic hardware.

It is currently only implemented and tested with the following environment:
\begin{table}[H]
  \centering
  \begin{tabular}{|l|l|}
    \hline
    Item          & Version \\
    \hline
    Hardware      & Cyclone V SX SoC development board \\
    HPS System    & Ubuntu 16.04.6 LTS (GNU/Linux 3.10.31 armv7l)\\
    HPS Python    & 2.7.12 \\
    Quartus Prime & 16.0.0.211 \\
    \hline
  \end{tabular}
  \caption{Tested Environment}
\end{table}

To download this implementation, use

\texttt{git clone https://github.com/MerelyLogical/arithmetic-testbench.git}

It includes both hardware and software files.

\section{Configuring Hardware}
The hardware needs to be configured to fit the design before compiled and uploaded onto the development board.
This guide provides the steps to this configuration with Quartus' GUI in a beginner-friendly way.
It should be noted that this configuration can also be done by editing the \texttt{.qsys} file directly.
The summary of this section is to first integrate the user's design into the Qsys system as a component.
Then the system is compiled into synthesisable HDL.
The whole system is then synthesised fitted, and then assembled into an \texttt{.sof} file.
In order to upload this to the FPGA through the HPS later, it is converted to an \texttt{.rbf}, but this is optional if a different method of programming the FPGA is desired.

\begin{enumerate}
  \item Open Quartus 16.0
  \item File $\rightarrow$ Open Project... $\rightarrow$ \texttt{cy5-systest.qpf}.
  \item Tools $\rightarrow$ Qsys
  \item File $\rightarrow$ Open... $\rightarrow$ \texttt{soc\_system.qsys}.
        (The open dialogue may pop-up automatically on start up of Qsys.)
  \item In the IP Catalog window, click New..., which will bring out the component editor.
        This turns your designed components into Qsys components, which will then be integrated into the Qsys system.
  \item Enter basic information of your design in the \textit{Component} tab of the component editor.
  \item In the \textit{Files} tab, add your design files, select the top-level module and click on \textit{Analyze Synthesis Files}.
  \item Module parameters and be set in the \textit{Parameters} tab.
  \item The \textit{Signals \& Interfaces} tab should have detected all ports of your design.
        First ensure that the clock and the reset inputs are detected and categorised in the interface list correctly.
        Then to allow connection to the rest of the testbench, move the two inputs and the one output of your design into a \textit{Conduit} interface.
        Associate the clock and the reset signal to the conduit.
        Then name the two inputs a and b, and name the output as out.
  \item Finish... $\rightarrow$ Yes, Save $\rightarrow$ Yes, save before refresh.
  \item The created component should now show up in the \textit{IP Catalog}.
  \item Add the component to the system, enter the desired parameter values before clicking on \textit{Finish}.
  \item Connect the signals of your design using the \textit{System Contents} window of Qsys.
        The clock should be connnected to the \texttt{outclk0} signal of the component \texttt{pll\_dut}.
        The reset signal should be connected to the \texttt{clk\_reset} signal of the component \texttt{clk\_ref\_50M}.
        The conduit should be connected to the opposing conduit in the \texttt{test\_wrapper} module.
  \item Click on the \texttt{test\_wrapper} module and configure its parameters where necessary.
        \texttt{NUM\_SUB\_MON} determines how many sub-monitors are spawned to run in parallel.
        \texttt{WIDTH} should match the width of your design's I/O width.
  \item Fix any outstanding errors in the \texttt{Messages} window if preset.
  \item Click on Generate HDL...
  \item Ensure \textit{Create HDL design files for synthesis} is not on none.
  \item Generate the HDL files.
  \item Click on Finish to close Qsys.
  \item Quartus should prompt you to add the \texttt{.qip} file and the \texttt{.sip} file.
        Follow them by going to Project $\rightarrow$ Add/Remove Files to Project...
  \item After making sure that the two Qsys files are included, start the compilation by going to Processing $\rightarrow$ Start Compilation.
  \item By the end of the compilation, the assembler should have produced \texttt{output\_files/cy5-systest.sof}.
        We need this in a different format, so go to File $\rightarrow$ Convert Programming Files...
  \item Select \texttt{.rbf} as the output type.
  \item Click on the only entry in the input files table, and click on Add File... to add the produced \texttt{.sof} file.
  \item The hardware setup is completed after obtaining the \texttt{.rbf} file.
\end{enumerate}

\section{Setting up the Board}
By the end of this sections, we should have the FPGA programmed and the software uploaded to the HPS, ready to start running the tests.
In this guide we will be using rsync to upload the \texttt{/scripts} folder and the \texttt{.rbf} file into the board, and programming the FPGA from the HPS, but other methods are equally as valid.

\begin{enumerate}
  \item Copy the \texttt{.rbf} file into \texttt{/scripts} and upload it with \texttt{rsync}.

  \texttt{rsync -rtuv ./ root@ee-cy5soc3.ee.ic.ac.uk:/home/\line(1,0){30}/scripts}
  \item Connect to the board.
        (This can be done by using PuTTY on Windows or a ssh command on Linux.)

  \texttt{ssh root@ee-cy5soc3.ee.ic.ac.uk}
  \item Program the FPGA.

  \texttt{cd /home/\line(1,0){30}/scripts}

  \texttt{./program\_fpga.sh output\_file.rbf}
\end{enumerate}


\section{Running Tests}

\texttt{python -O ./run\_test.py [file]}

If no file is provided, the software will enter a REPL, where the commands available are as follows.
If a file is provided, the software will read each line and execute the commands with the same syntax requirements, except for \texttt{exit} which will not work in a file.

\begin{table}[H]
  \centering
  \begin{tabular}{|>{\ttfamily}p{11em}|p{\dimexpr\textwidth-18em}|}
    \hline
    \textrm{Command}   & Explanation \\
    \hline
    reset              & Resets the system and test results. \\
    version            & Prints the system version. \\
    freq <speed>       & Sets the clock speed to the specified value in MHz. Prints the actual frequency configured. \\
    mode <m|a>         & Choose between \underline{m}anual and \underline{a}uto test mode. \\
    manual <a|b> <hex>  & Give input in manual mode. \\
    bitset <a|b> <hex>  & Force bits to be 1 in auto mode. \\
    bitclr <a|b> <hex>  & Force bits to be 0 in auto mode. \\
    run <time>         & Runs the test for specified duration in ms. Prints the results at the end of the test. \\
    exit               & Exits the REPL. \\
    \hline
  \end{tabular}
  \caption{Commands accepted in test REPL}
\end{table}

Notes:
\begin{itemize}
  \item Arguments in angle brackets are required.
        Arguments in square brackets are optional.
        Vertical bars separate all possible options.
  \item Frequency configuration is done by a PLL which has limited granularities.
        As such the actual frequency may differ from the desired frequency.
  \item Argument \texttt{<a|b>} is used to select which input this command will apply to.
  \item \texttt{<hex>} needs to be in the format \texttt{\textasciicircum(0x)?[0-9a-fA-F]\{1,8\}\$}.
        In words, it takes 1 to 8 digits of case-insensitive hexadecimal digits.
        No other characters including space or underscore are allowed, base prefix \texttt{0x} is unnecessary but allowed.
  \item Under the current hardware environment, a safe range for \texttt{<speed>} in MHz is from 50 to 400.
\end{itemize}
