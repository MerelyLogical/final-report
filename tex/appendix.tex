\chapter{User Guide Provided to Test Volunteers}

\begin{table}[H]
  \centering
  \begin{tabular}{|>{\ttfamily}p{11em}|p{\dimexpr\textwidth-18em}|}
    \hline
    \textrm{Command}   & Explanation \\
    \hline
    reset              & Resets the system and test results. \\
    version            & Prints the system version. \\
    freq <speed>       & Sets the clock speed to the specified value in MHz. Prints the actual frequency configured. \\
    mode <m|a>         & Choose between \underline{m}anual and \underline{a}uto test mode. \\
    manual <a|b> <hex>  & Give input in manual mode. \\
    bitset <a|b> <hex>  & Force bits to be 1 in auto mode. \\
    bitclr <a|b> <hex>  & Force bits to be 0 in auto mode. \\
    run <time>         & Runs the test for specified duration in ms. Prints the results at the end of the test. \\
    \hline
  \end{tabular}
  \caption{Commands accepted in test REPL}
\end{table}

Notes:
\begin{itemize}
  \item Arguments in angle brackets are required. Arguments in square brackets are optional.
  \item Frequency configuration is done by a PLL which has limited granularities. As such the actual frequency may differ from the desired frequency.
  \item Argument \texttt{<a|b>} is used to select which input this command will apply to.
  \item \texttt{<hex>} needs to be in the format \texttt{[0-9a-fA-F]\{1,8\}}. In words, it takes 1 to 8 digits of case-insensitive hexadecimal digits. No other characters including space or underscore is allowed, base prefixes such as \texttt{0x} are unnecessary and disallowed.
  \item Under the current hardware environment, a safe range for \texttt{<speed>} in MHz is from 50 to 400.
\end{itemize}


\chapter{Raw Data}

\textcolor{red}{+Code Snippets?}
