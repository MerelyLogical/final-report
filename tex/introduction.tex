\section{Introduction}

With the right number representation system, it is possible to perform arithmetic operations MSD first.
Consequently, these online arithmetic operators are attractive for hardware implementation in both serial and parallel forms.
When computing digits serially, they can be chained such that subsequent operations begin before the preceding ones complete.
Parallel implementations tend to be most sensitive to failure in their LSDs, making them more friendly to overclocking than their LSD first counterparts, for which the opposite is true.

In the past, online operators have typically been implemented in binary.
Although Radix-2 modules are the simplest to design and has the shortest cycle time per digit, it has the highest online delay and requires the largest number of cycles to complete calculations~\cite{Tenca1}.
As such, the choice of binary is not absolute.
In this project, I will explore high-radix online operators, investigating their suitability for FPGA implementation and examining the resultant tradeoffs between performance, area and power.
