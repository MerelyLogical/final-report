% This project concerns the research and design of a driving behaviour model that could identify the user from its inputs and predict his next actions for a small interval of time in the future.
% The main emphasis of this project is on implementing the necessary tools for capturing and processing the data in order to be appropriate for machine learning techniques. Using a virtual simulator, multiple users were recorded, while driving several laps of different tracks.
% The user’s actions recorded were the angles turned on the steering wheel, strength applied to the gas and brake pedals, the gears involved and speed of the car.
% By experimenting with numerous algorithms and approaching different models, it was observed that driving is a sequential model that depends on the previous and current driver’s actions, the car’s response and the environment.
% Random Forests, Decision trees and Hidden Markov model algorithms have been used with both independent windows of data and sequences of data for predicting, with high accuracy, the identity of the driver.
% Feature extraction as well as feature selection algorithms were introduced to increase the performance of the classifiers.
% Experiments concerning sequences of feature vectors resulted on more accurate identification of the users.

\begin{abstract}
This project concerns the research and design of a driving behaviour model that could identify the user from its inputs and predict his next actions for a small interval of time in the future.
The main emphasis of this project is on implementing the necessary tools for capturing and processing the data in order to be appropriate for machine learning techniques. Using a virtual simulator, multiple users were recorded, while driving several laps of different tracks.
The user’s actions recorded were the angles turned on the steering wheel, strength applied to the gas and brake pedals, the gears involved and speed of the car.
By experimenting with numerous algorithms and approaching different models, it was observed that driving is a sequential model that depends on the previous and current driver’s actions, the car’s response and the environment.
Random Forests, Decision trees and Hidden Markov model algorithms have been used with both independent windows of data and sequences of data for predicting, with high accuracy, the identity of the driver.
Feature extraction as well as feature selection algorithms were introduced to increase the performance of the classifiers.
Experiments concerning sequences of feature vectors resulted on more accurate identification of the users.
\end{abstract}