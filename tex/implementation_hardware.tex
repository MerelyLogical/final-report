\chapter{Hardware Implementation}

\section{Randomiser}

\begin{figure}[H]
  \centering
  \begin{tikzpicture}
  [
    x=1em, y=1em,
    block/.style =
      {draw, rectangle, align=center, minimum width=4em, minimum height=6em},
    iarrow/.style =
      {<-, >={Stealth}, font=\ttfamily},
    oarrow/.style =
      {->, >={Stealth}, font=\ttfamily}
  ]

\node[block, label=above:Randomiser] (r) at (0,0) {};

\draw [iarrow] ($(r.west)+(0,2)$) -- ++(left:3) node[left] {clk};
\draw [iarrow] ($(r.west)+(0,1)$) -- ++(left:3) node[left] {reset};
\draw [iarrow] ($(r.west)-(0,1)$) -- ++(left:3) node[left] {enable};
\draw [iarrow] ($(r.west)-(0,2)$) -- ++(left:3) node[left] {initial};

\draw [oarrow] ($(r.east)+(0,2)$) -- ++(right:3) node[right] {out};

\end{tikzpicture}
  \caption{Randomiser Block Diagram}
  \label{RandomiserBlk}
\end{figure}

Implementing the randomiser is straight forward.
A possible set of taps for a 32-bit Galois LFSR is [32, 30, 26, 25].
Referring back at Figure~\ref{GalLFSR} on page~\pageref{GalLFSR},
the logic is to XOR the bits left of the taps with bit 0, and simple right shift for all other bits.
For driver to control the randomiser, an enable signal and an initial signal is added as input in addition to clock and reset.
The initial signal seeds the LFSR.

\section{Driver}

\begin{figure}[H]
  \centering
  \begin{tikzpicture}
  [
    x=1em, y=1em,
    block/.style =
      {draw, rectangle, align=center, minimum width=4em, minimum height=10em},
    sarrow/.style =
      {>={Stealth}, font=\ttfamily},
    darrow/.style =
      {double distance=1.5pt, >={Stealth}, font=\ttfamily}
  ]

\node[block, label=above:Driver] (r) at (0,0) {};

\draw [<-, sarrow] ($(r.west)+(0,4.5)$) -- ++(left:3) node[left] {clk};
\draw [<-, sarrow] ($(r.west)+(0,3.5)$) -- ++(left:3) node[left] {reset};

\draw [<-, sarrow] ($(r.west)+(0,1.5)$) -- ++(left:3) node[left] {f\_select};
\draw [<-, darrow] ($(r.west)+(0,0.5)$) -- ++(left:3) node[left] {f\_manual};
\draw [<-, sarrow] ($(r.west)-(0,0.5)$) -- ++(left:3) node[left] {f\_bitset};
\draw [<-, sarrow] ($(r.west)-(0,1.5)$) -- ++(left:3) node[left] {f\_bitclr};
\draw [<-, darrow] ($(r.west)-(0,2.5)$) -- ++(left:3) node[left] {rand\_*};

\draw [<-, darrow] ($(r.west)-(0,4.5)$) -- ++(left:3) node[left] {dut\_out};

\draw [->, sarrow] ($(r.east)-(0,4.5)$) -- ++(right:3) node[right] {dut\_delay};

\draw [->, darrow] ($(r.east)-(0,1.5)$) -- ++(right:3) node[right] {drive\_*};
\draw [->, darrow] ($(r.east)-(0,2.5)$) -- ++(right:3) node[right] {drive\_delayed\_*};
\end{tikzpicture}
  \caption{Driver Block Diagram}
  \label{DriverBlk}
\end{figure}

The filter select signal \texttt{f\_select} selects the mode of operation of the driver.
When it is set, the driver will read from \texttt{f\_manual} and feed them to the output.
Otherwise, the driver will take the output of the randomisers at \texttt{rand\_*}, set and clearing specific bits according to \texttt{f\_bitset} and \texttt{f\_bitclr}.

The output is immediately sent to the DUT from the ports \texttt{drive\_*}.
The output is also delayed for a number of cycles before being sent to the monitor from the ports \texttt{drive\_delayed\_*}.
This delay is known and thus can be configured by the user before compiling the testbench.

\section{Monitor}
\begin{figure}[H]
  \centering
  \begin{tikzpicture}
  [
    x=1em, y=1em,
    block/.style =
      {draw, rectangle, align=center, minimum width=4em, minimum height=5em},
    sarrow/.style =
      {>={Stealth}, font=\ttfamily},
    darrow/.style =
      {double distance=1.5pt, >={Stealth}, font=\ttfamily}
  ]

\node[block, label=above:Monitor] (r) at (0,0) {};
\node[draw, opacity=0, rectangle, minimum width=30em] () at (0,0) {};

\draw [<-, sarrow] ($(r.west)+(0,2)$) -- ++(left:3) node[left] {clk};
\draw [<-, sarrow] ($(r.west)+(0,1)$) -- ++(left:3) node[left] {reset};

\draw [<-, darrow] ($(r.west)-(0,1)$) -- ++(left:3) node[left] {drive\_mon\_*};
\draw [<-, darrow] ($(r.west)-(0,2)$) -- ++(left:3) node[left] {dut\_out};

\draw [->, sarrow] ($(r.east)-(0,1)$) -- ++(right:3) node[right] {mon\_ready};
\draw [->, darrow] ($(r.east)-(0,2)$) -- ++(right:3) node[right] {diff};

\end{tikzpicture}
  \caption{Monitor Block Diagram}
  \label{MonitorBlk}
\end{figure}

\subsection{Sub-monitors}

\begin{figure}[H]
  \centering
  \begin{tikzpicture}
  [
    x=1em, y=1em,
    block/.style =
      {draw, rectangle, align=center, minimum width=4em, minimum height=5em},
    sarrow/.style =
      {>={Stealth}, font=\ttfamily},
    darrow/.style =
      {double distance=1.5pt, >={Stealth}, font=\ttfamily}
  ]

\node[block, label=above:Sub-monitor] (r) at (0,0) {};

\draw [<-, sarrow] ($(r.west)+(0,2)$) -- ++(left:3) node[left] {clk};
\draw [<-, sarrow] ($(r.west)+(0,1)$) -- ++(left:3) node[left] {reset};

\draw [<-, darrow] ($(r.west)-(0,1)$) -- ++(left:3) node[left] {i*};
\draw [<-, darrow] ($(r.west)-(0,2)$) -- ++(left:3) node[left] {dut\_o};

\draw [->, darrow] ($(r.east)-(0,1)$) -- ++(right:3) node[right] {mon\_o};
\draw [->, darrow] ($(r.east)-(0,2)$) -- ++(right:3) node[right] {dtm\_o};

\end{tikzpicture}
  \caption{Sub-monitor Block Diagram}
  \label{SubmonBlk}
\end{figure}

\textcolor{red}{2 Pages here, waveform diagram not correct}
\begin{figure}[H]
  \centering
  \begin{tikztimingtable}
    [
      xscale=4,
      timing/d/background/.style={fill=white},
      timing/font=\ttfamily
    ]
    dist\_ctr & D{8} 2{D{1}D{2}D{4}D{8}} D{1}D{2}         \\
    a & U D{AA}D{AB}D{AC}D{AD}D{AE}D{AF}D{AG}D{AH}D{AI} U \\
    b & U D{BA}D{BB}D{BC}D{BD}D{BE}D{BF}D{BG}D{BH}D{BI} U \\
    s & 3U D{SA}D{SB}D{SC}D{SD}D{[red]SK}D{SF}D{SG}D      \\
    \\
    dist\_ctr  [0] & L 2{H 3L} HL \\
    a\_mon     [0] & U 4D{AA} 4D{AE} 2D{AI} \\
    b\_mon     [0] & U 4D{BA} 4D{BE} 2D{BI} \\
    s\_mon     [0] & 2U 4D{SA} 4D{SE} D \\
    sub\_event [0] & 8LH2L \\
  \extracode
    % Add vertical lines in two colors
    \begin{pgfonlayer}{background}
      \begin{scope}[semitransparent,semithick]
        \vertlines{1,2,...,10}
      \end{scope}
    \end{pgfonlayer}
  \end{tikztimingtable}
  \caption{Distributed Monitoring System}
  \label{DisMon}
\end{figure}

\section{Scoreboard}

\begin{figure}[H]
  \centering
  \begin{tikzpicture}
  [
    x=1em, y=1em,
    block/.style =
      {draw, rectangle, align=center, minimum width=4em, minimum height=7em},
    sarrow/.style =
      {>={Stealth}, font=\ttfamily},
    darrow/.style =
      {double distance=1.5pt, >={Stealth}, font=\ttfamily}
  ]

\node[block, label=above:Scoreboard] (r) at (0,0) {};

\draw [<-, sarrow] ($(r.west)+(0,3)$) -- ++(left:3) node[left] {clk};
\draw [<-, sarrow] ($(r.west)+(0,2)$) -- ++(left:3) node[left] {reset};

\draw [<-, sarrow] ($(r.west)+(0,0)$) -- ++(left:3) node[left] {freeze};
\draw [<-, sarrow] ($(r.west)-(0,1)$) -- ++(left:3) node[left] {mon\_ready};

\draw [<-, darrow] ($(r.west)-(0,3)$) -- ++(left:3) node[left] {diff};

\draw [->, sarrow] ($(r.east)+(0,1)$) -- ++(right:3) node[right] {maxacc};
\draw [->, sarrow] ($(r.east)+(0,0)$) -- ++(right:3) node[right] {minacc};

\draw [->, sarrow] ($(r.east)-(0,2)$) -- ++(right:3) node[right] {data\_ctr};
\draw [->, sarrow] ($(r.east)-(0,3)$) -- ++(right:3) node[right] {error\_ctr};

\end{tikzpicture}
  \caption{Scoreboard Block Diagram}
  \label{ScoreboardBlk}
\end{figure}
