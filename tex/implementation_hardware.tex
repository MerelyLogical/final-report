\chapter{Hardware Implementation}

\section{Randomiser}

\begin{figure}[H]
  \centering
  \begin{tikzpicture}
  [
    x=1em, y=1em,
    block/.style =
      {draw, rectangle, align=center, minimum width=4em, minimum height=5em},
    sarrow/.style =
      {>={Stealth}, font=\ttfamily},
    darrow/.style =
      {double distance=1.5pt, >={Stealth}, font=\ttfamily}
  ]

\node[block, label=above:Randomiser] (r) at (0,0) {};

\draw [<-, sarrow] ($(r.west)+(0,2)$) -- ++(left:3) node[left] {clk};
\draw [<-, sarrow] ($(r.west)+(0,1)$) -- ++(left:3) node[left] {reset};

\draw [<-, sarrow] ($(r.west)-(0,1)$) -- ++(left:3) node[left] {enable};
\draw [<-, sarrow] ($(r.west)-(0,2)$) -- ++(left:3) node[left] {initial};

\draw [->, darrow] ($(r.east)-(0,2)$) -- ++(right:3) node[right] {out};

\end{tikzpicture}
  \caption{Randomiser Block Diagram}
  \label{RandomiserBlk}
\end{figure}

Implementing the randomiser is straight forward.
A possible set of taps for a 32-bit Galois LFSR is [32, 30, 26, 25].
Referring back at Figure~\ref{GalLFSR} on page~\pageref{GalLFSR},
the logic is to XOR the bits left of the taps with bit 0, and simple right shift for all other bits.
For driver to control the randomiser, an enable signal and an initial signal is added as input in addition to clock and reset.
The \texttt{initial} signal seeds the LFSR.

\section{Driver}

\begin{figure}[H]
  \centering
  \begin{tikzpicture}
  [
    x=1em, y=1em,
    block/.style =
      {draw, rectangle, align=center, minimum width=4em, minimum height=10em},
    sarrow/.style =
      {>={Stealth}, font=\ttfamily},
    darrow/.style =
      {double distance=1.5pt, >={Stealth}, font=\ttfamily}
  ]

\node[block, label=above:Driver] (r) at (0,0) {};
\node[draw, opacity=0, rectangle, minimum width=30em] () at (0,0) {};

\draw [<-, sarrow] ($(r.west)+(0,4.5)$) -- ++(left:3) node[left] {clk};
\draw [<-, sarrow] ($(r.west)+(0,3.5)$) -- ++(left:3) node[left] {reset};

\draw [<-, sarrow] ($(r.west)+(0,1.5)$) -- ++(left:3) node[left] {f\_select};
\draw [<-, darrow] ($(r.west)+(0,0.5)$) -- ++(left:3) node[left] {f\_manual};
\draw [<-, sarrow] ($(r.west)-(0,0.5)$) -- ++(left:3) node[left] {f\_bitset};
\draw [<-, sarrow] ($(r.west)-(0,1.5)$) -- ++(left:3) node[left] {f\_bitclr};
\draw [<-, darrow] ($(r.west)-(0,2.5)$) -- ++(left:3) node[left] {rand\_*};

\draw [<-, darrow] ($(r.west)-(0,4.5)$) -- ++(left:3) node[left] {dut\_out};

\draw [->, sarrow] ($(r.east)-(0,1.5)$) -- ++(right:3) node[right] {dut\_delay};

\draw [->, darrow] ($(r.east)-(0,3.5)$) -- ++(right:3) node[right] {drive\_*};
\draw [->, darrow] ($(r.east)-(0,4.5)$) -- ++(right:3) node[right] {drive\_delayed\_*};
\end{tikzpicture}
  \caption{Driver Block Diagram}
  \label{DriverBlk}
\end{figure}

The filter select signal \texttt{f\_select} selects the mode of operation of the driver.
When it is set, the driver will read from \texttt{f\_manual} and feed them to the output.
Otherwise, the driver will take the output of the randomisers at \texttt{rand\_*}, set and clearing specific bits according to \texttt{f\_bitset} and \texttt{f\_bitclr}.

The output is immediately sent to the DUT from the ports \texttt{drive\_dut\_*}.
The output is also delayed for a number of cycles before being sent to the monitor from the ports \texttt{drive\_mon\_*}.
This delay is known and thus can be configured by the user before compiling the testbench.

\begin{figure}[H]
  \centering
  \begin{tikztimingtable}
  [
    xscale=4,
    timing/d/background/.style={fill=white},
    timing/font=\ttfamily
  ]
  clk        & h 21{c} \\
  rand       & D{a4fe}D{527f}D{a93f}D{d49f}D{ea4f}D{7527}D{ba93}D{5d49}D{2ea4}D{1752}D{8ba9} \\
  f\_bitset  & 4D{0000} 7D{0004} \\
  f\_bitclr  & 3D{0000} 3D{f000} 5D{0000} \\
  drive\_dut & U D{a4fe}D{527f}D{a93f}D{049f}D{0a4f}D{0527}D{ba97}D{5d4d}D{1756}D{8bad} \\
  drive\_mon & 3U D{a4fe}D{527f}D{a93f}D{049f}D{0a4f}D{0527}D{ba97}D{5d4d} \\
  dut\_out   & 3U D{a4fe}D{527f}D{a93f}D{049f}D{0a4f}D{0527}D{ba97}D{5d4d} \\
\extracode
  % Add vertical lines in two colors
  \begin{pgfonlayer}{background}
    \begin{scope}[semitransparent,semithick]
      \vertlines{1,2,...,10}
    \end{scope}
  \end{pgfonlayer}
\end{tikztimingtable}
  \caption{Driver Waveform}
  \label{DriveWave}
\end{figure}

Figure \ref{DriveWave} shows how the waveform of the implemented driver looks like when operating in auto mode.
It should be stated that all waveforms in this report has been obtained from simulation, and are verified to be correct.
For clarity, unnecessary signals and signal values are omitted.

In the example waveform, we assume the design is a simple 1 input 1 output module where the input is passed to the output after 2 cycles.
The driver passed \texttt{rand} to \texttt{drive\_dut} after a cycle.
When \texttt{f\_bitclr} is \texttt{f000}, the top 4 bits of the output are set to 0, and when \texttt{f\_bitset} is \texttt{0004}, bit 2 of the output is set to 1.
It should be noted, that if the same bit is set and cleared by the user, \texttt{f\_bitclr} takes priority.
This is an arbitrary choice, and is noted in the user guide.

The driver delays the output to monitor by 2 cycles, and we can see that this aligns it the output from the DUT.

\section{Monitor}
\begin{figure}[H]
  \centering
  \begin{tikzpicture}
  [
    x=1em, y=1em,
    block/.style =
      {draw, rectangle, align=center, minimum width=4em, minimum height=5em},
    sarrow/.style =
      {>={Stealth}, font=\ttfamily},
    darrow/.style =
      {double distance=1.5pt, >={Stealth}, font=\ttfamily}
  ]

\node[block, label=above:Monitor] (r) at (0,0) {};
\node[draw, opacity=0, rectangle, minimum width=30em] () at (0,0) {};

\draw [<-, sarrow] ($(r.west)+(0,2)$) -- ++(left:3) node[left] {clk};
\draw [<-, sarrow] ($(r.west)+(0,1)$) -- ++(left:3) node[left] {reset};

\draw [<-, darrow] ($(r.west)-(0,1)$) -- ++(left:3) node[left] {drive\_mon\_*};
\draw [<-, darrow] ($(r.west)-(0,2)$) -- ++(left:3) node[left] {dut\_out};

\draw [->, sarrow] ($(r.east)-(0,1)$) -- ++(right:3) node[right] {mon\_ready};
\draw [->, darrow] ($(r.east)-(0,2)$) -- ++(right:3) node[right] {diff};

\end{tikzpicture}
  \caption{Monitor Block Diagram}
  \label{MonitorBlk}
\end{figure}

The monitor takes DUT inputs from the driver, and distributes them to a few sub-monitors.
Each sub-monitor containing a reference design then produces the correct results \texttt{mon\_out} from the inputs with a relaxed time budget.
The monitor then checks the difference between the reference output, \texttt{mon\_out}, and the DUT output, \texttt{dut\_out} with XOR gates.
This results in \texttt{diff}, where each bit set to 1 indicates a wrong bit in the DUT output.
\texttt{mon\_ready} will be set after the distributor has completed an entire round, where the first meaningful \texttt{diff} value becomes available.

As the number of sub-monitors and the width of the tested unit are parametrised, the design of the distributor was not straightforward.
An one-hot counter is set up to determined the currently active sub-monitor.
Since the sub-monitors are clocked \texttt{NUM\_SUB\_MON} times slower, an array of \texttt{NUM\_SUB\_MON} registers each of size \texttt{WIDTH} is also created for each input or output of the design.
These serves as the interface between the sub-monitors and the rest of the design.

\subsection{Sub-monitors}

\begin{figure}[H]
  \centering
  \begin{tikzpicture}
  [
    x=1em, y=1em,
    block/.style =
      {draw, rectangle, align=center, minimum width=4em, minimum height=5em},
    sarrow/.style =
      {>={Stealth}, font=\ttfamily},
    darrow/.style =
      {double distance=1.5pt, >={Stealth}, font=\ttfamily}
  ]

\node[block, label=above:Sub-monitor] (r) at (0,0) {};
\node[draw, opacity=0, rectangle, minimum width=30em] () at (0,0) {};

\draw [<-, sarrow] ($(r.west)+(0,2)$) -- ++(left:3) node[left] {clk};
\draw [<-, sarrow] ($(r.west)+(0,1)$) -- ++(left:3) node[left] {reset};

\draw [<-, darrow] ($(r.west)-(0,1)$) -- ++(left:3) node[left] {i*};
\draw [<-, darrow] ($(r.west)-(0,2)$) -- ++(left:3) node[left] {dut\_out};

\draw [->, darrow] ($(r.east)-(0,1)$) -- ++(right:3) node[right] {mon\_out};
\draw [->, darrow] ($(r.east)-(0,2)$) -- ++(right:3) node[right] {dtm\_out};

\end{tikzpicture}
  \caption{Sub-monitor Block Diagram}
  \label{SubmonBlk}
\end{figure}

The current sub-monitor is a part of the monitor module in the architecture design that interfaces with the reference module.
In addition to connecting with the reference inputs and outputs, the sub-monitors also handles the delay of the reference module.
This is due to the highly parametrised nature of the monitor module, which made it rather complex and inflexible to the addition of more features.

As such, there is an extra signal of \texttt{dtm\_out}, which has the same value as \texttt{dut\_out}, but delayed by the number of cycles that the reference design needs to complete its operation, thus aligning with \texttt{mon\_out}.
The other signals are directly connected to the reference module.

\begin{figure}[H]
  \centering
  \begin{tikztimingtable}
  [
    xscale=4,
    timing/d/background/.style={fill=white},
    timing/font=\ttfamily
  ]
  clk           & h 19{c} \\
  dist\_ctr     & D{100} 3{D{001}D{010}D{100}} \\
  drive\_mon\_a & U D{0123} 2U D{3210} 2U D{0213} 2U \\
  drive\_mon\_b & U D{4567} 2U D{7654} 2U D{4657} 2U \\
  dut\_out      & U D{468a} 2U D{a861} 2U D{486a} 2U \\
  clk\_sub  [0] & 2L 2{2{c} 2L} 2{c} L \\
  a         [0] & 2U 3D{0123} 3D{3210} 2D{0213} \\
  b         [0] & 2U 3D{4567} 3D{7654} 2D{4657} \\
  dut\_o    [0] & 2U 3D{468a} 3D{a861} 2D{486a} \\
  mon\_o    [0] & 2U 3D{468a} 3D{a864} 2D{486a} \\
  diff          & 2U 3U D{0000} 2U D{0005} U \\
\extracode
  % Add vertical lines in two colors
  \begin{pgfonlayer}{background}
    \begin{scope}[semitransparent,semithick]
      \vertlines{1,2,...,9}
    \end{scope}
  \end{pgfonlayer}
\end{tikztimingtable}
  \caption{Monitor Waveform}
  \label{MonitorWave}
\end{figure}

Figure \ref{MonitorWave} shows the waveform of a monitor with \texttt{NUM\_SUB\_MON} as 3.
The reference adder takes 1 cycle to complete, but it must be clocked at a frequency slower than that of the adder DUT.
With 3 sub-monitors, the width of \texttt{dist\_ctr} is 3, and its lowest bit corresponds to sub-monitor 0, which is shown in detail in this figure.
\texttt{clk\_sub} is the clock driving the sub-monitor, which is made by masking the DUT clock with a delayed version of \texttt{dist\_ctr}.
As it ticks, the I/O values are copied into the register arrays and held for 3 cycles.
Within this time, the reference design completes its operation, fixing the value on \texttt{mon\_o}.
When the cycle of \texttt{dist\_ctr} goes a full cycle and the \texttt{sub\_clk} ticks again, this value is collected back and XOR'ed to form the final output of the monitor, \texttt{diff}.
In the example of Figure \ref{MonitorWave}, the second result was an error on the DUT, as it gave \texttt{a861} while the reference answer was \texttt{a864}.
This means bit 0 and 2 were different, and \texttt{diff} is thus \texttt{0005}.

The other sub-monitors all work identically but each 1 DUT cycle later the last.
This allows the reference design to run slower than the DUT, to still provide a constant stream of \texttt{diff} at the monitor output as designed.

\section{Scoreboard}

\begin{figure}[H]
  \centering
  \begin{tikzpicture}
  [
    x=1em, y=1em,
    block/.style =
      {draw, rectangle, align=center, minimum width=4em, minimum height=7em},
    sarrow/.style =
      {>={Stealth}, font=\ttfamily},
    darrow/.style =
      {double distance=1.5pt, >={Stealth}, font=\ttfamily}
  ]

\node[block, label=above:Scoreboard] (r) at (0,0) {};
\node[draw, opacity=0, rectangle, minimum width=30em] () at (0,0) {};

\draw [<-, sarrow] ($(r.west)+(0,3)$) -- ++(left:3) node[left] {clk};
\draw [<-, sarrow] ($(r.west)+(0,2)$) -- ++(left:3) node[left] {reset};

\draw [<-, sarrow] ($(r.west)+(0,0)$) -- ++(left:3) node[left] {freeze};
\draw [<-, sarrow] ($(r.west)-(0,1)$) -- ++(left:3) node[left] {mon\_ready};

\draw [<-, darrow] ($(r.west)-(0,3)$) -- ++(left:3) node[left] {diff};

\draw [->, sarrow] ($(r.east)+(0,1)$) -- ++(right:3) node[right] {maxacc};
\draw [->, sarrow] ($(r.east)+(0,0)$) -- ++(right:3) node[right] {minacc};

\draw [->, sarrow] ($(r.east)-(0,2)$) -- ++(right:3) node[right] {data\_ctr};
\draw [->, sarrow] ($(r.east)-(0,3)$) -- ++(right:3) node[right] {error\_ctr};

\end{tikzpicture}
  \caption{Scoreboard Block Diagram}
  \label{ScoreboardBlk}
\end{figure}

This scoreboard tracks the number of valid test points going through with \texttt{data\_ctr} and the number of errors within them with \texttt{error\_ctr}.
The external input \texttt{freeze} is exposed to the software to stop all counting in the scoreboard.
As the current HPS-FPGA bridge set up only allows sequential reads to the FPGA registers, it is necessary to ensure the values do not change within a single set of read commands from the software.

Another implication of the current bridge set up is that there is no simple way of getting all the \texttt{diff} values out to the HPS for statistics calculations.
This is a limitation of the current implementation, and will be discussed in the Further work section of this report.
Therefore, the hardware will have to do some simple statistics.

Since we are interested in how the accuracy of the DUT degrades as the frequency increases, two signals are created to record the maximum and the minimum accuracy of the DUT output.
Calculating the accuracy with the \texttt{diff} signal means counting the number of leading zeros (\texttt{CLZ}), as zeros indicate the correct bits.
As the current implementation is limited with a maximum width of 32, the easiest way of doing \texttt{CLZ} fast is by padding zeros after the number to 32 bits, and then use a large lookup table with don't cares.
This accuracy signal is named \texttt{acc}.

To keep track of the minimum accuracy, register \texttt{minacc} is first initialised to the maximum, and then for each smaller value observed, it will take on its smaller value.
The comparison logic here is relatively expensive in this fast testbench design.
As such, there is great incentive in the future to build a better communication method to allow offloading these operations to the HPS.

\begin{figure}[H]
  \centering
  \begin{tikztimingtable}
  [
    xscale=4,
    timing/d/background/.style={fill=white},
    timing/font=\ttfamily
  ]
  clk        & h 19{c} \\
  freeze     & 8L 2H \\
  mon\_ready & 2L 8H \\
  diff       & D{000f}D{0001}D{0000}D{0001}D{000c}2D{0000}3D{0001} \\
  data\_ctr  & 3D{0} 1R 6{Q} D{0} \\
  error\_ctr & 4D{0} D{1} 3D{2} D{3} D{4} \\
  maxacc     & 3D{0} 7D{16} \\
  minacc     & 3D{16} D{15} 6D{12} \\
\extracode
  % Add vertical lines in two colors
  \begin{pgfonlayer}{background}
    \begin{scope}[semitransparent,semithick]
      \vertlines{1,2,...,9}
    \end{scope}
  \end{pgfonlayer}
\end{tikztimingtable}
  \caption{Scoreboard Waveform}
  \label{ScoreboardWave}
\end{figure}

Figure \ref{ScoreboardWave} shows a example waveform.
The counters and the extrema trackers changes value only if we have \texttt{mon\_ready \&\& !freeze}.


\section{Summary}

Figure \ref{DBlock} provides a detailed look at how the individual hardware components are wired together to form the \texttt{testbench} module.
The inputs and outputs of this module is contained within \texttt{test\_wrapper}, which handles the AXI communications when coupled with the \texttt{hps} module.
As discussed in the software implementation chapter, this allowed the software to access the testbench, completing the overall system implementation.

\begin{sidewaysfigure}
  \begin{figure}[H]
    \centering
    \begin{tikzpicture}
  [
    x=1em, y=1em,
    block/.style =
      {draw, rectangle, align=center, minimum width=4em, minimum height=6em},
    sarrow/.style =
      {>={Stealth}, font=\ttfamily},
    darrow/.style =
      {double distance=1.5pt, >={Stealth}, font=\ttfamily}
  ]
  \node [block, label=above:Randomiser]  at ( 8, 10)  (r) {};
  \node [block, label=above:Driver]      at (16, 10)  (d) {};
  \node [block, label=above:DUT]         at (24,  0)  (t) {};
  \node [block, label=above:Monitor]     at (32, 10)  (m) {};
  \node [block, label=below:Sub-monitor] at (36,  0)  (u) {};
  \node [block, label=above:Scoreboard]  at (40, 10)  (s) {};

  \draw [->, sarrow] (2,15) node[left] {enable}
                      -| ($(r.west)+(-1,2)$)
                      -- ($(r.west)+(0,2)$);
  \draw [->, sarrow] (2,16) node[left] {f\_*}
                      -| ($(d.west)+(-1,2)$)
                      -- ($(d.west)+(0,2)$);
  \draw [->, sarrow] (2,17) node[left] {freeze}
                      -| ($(s.west)+(-1,2)$)
                      -- ($(s.west)+(0,2)$);

  \draw [->, sarrow] ($(d.east)-(0,2)$)
                      -| ++(1,-2)
                      -- ++(-14,0)
                      |- node[left] {initial} ($(r.west)-(0,2)$);
  \draw [->, darrow] ($(r.east)+(0,0)$)
                      -- node[above] {rand\_*} ($(d.west)+(0,0)$);
  \draw [->, darrow] ($(d.east)+(0,1)$)
                      -- node[above] {drive\_mon\_*} ($(m.west)+(0,1)$);
  \draw [->, darrow] ($(d.east)-(0,1)$)
                      -- ++(2,0)
                      -- node[right,pos=0] {drive\_dut\_*} ($(t.west)+(-2,0)$)
                      -- ($(t.west)+(0,0)$);
  \draw [->, darrow] ($(t.east)-(0,0)$)
                      -- ++(2,0)
                      -- node[right,pos=0] {dut\_out} ($(m.west)-(2,1)$)
                      -- ($(m.west)-(0,1)$);
  \draw [->, darrow] ($(t.east)+(2,0)$)
                      |- ($(d.west)-(1,14)$)
                      |- ($(d.west)-(0,2)$);
  \draw [->, darrow] ($(m.south)-(1,0)$)
                      -- ++(0,-2.5)
                      -- node[below,pos=0.2] {mon\_i\_*} ($(u.north)-(1,-1.5)$)
                      -- ($(u.north)-(1,0)$);
  \draw [<-, darrow] ($(m.south)+(1,0)$)
                      -- ++(0,-1.5)
                      -- node[above,pos=0.8] {mon\_out} ($(u.north)+(1,2.5)$)
                      -- ($(u.north)+(1,0)$);
  \draw [->, darrow] ($(m.east)+(0,0)$)
                      -- node[above] {diff} ($(s.west)+(0,0)$);
  \draw [->, sarrow] ($(m.east)-(0,2)$)
                      -- node[above] {ready} ($(s.west)-(0,2)$);
  \draw [->, sarrow] ($(s.east)+(0,2)$)
                      -| ++(1,5)
                      -- (46,17) node[right] {results};

\end{tikzpicture}
    \caption{Block diagram of the implemented testbench}
    \label{DBlock}
  \end{figure}
\end{sidewaysfigure}

