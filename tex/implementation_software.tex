\chapter{Software Implementation}

\section{Accessing the FPGA}
The interfaces are mapped onto the physical memory, thus they can be accessed by opening \texttt{/dev/mem} with the \texttt{mmap} library in the Python test script.
The 4 bytes of binary data from the FPGA also needs to be packed or unpacked with the \texttt{struct} library, as they needs to be interpreted as unsigned long integers in little-endian.
The read and write function are defined in a class called \texttt{axi}.
The base of this function is provided to me by my supervisor.

Since the design is to abstract away direct interactions with the memory locations in the test wrapper, another class called the \texttt{wrapper} is created to serve as a collection of useful read and write functions.
Similarly, a class called \texttt{pll} is created to serve the same purpose, but for the PLL reconfiguration module.
The initial version of the \texttt{pll} class was provided to me by my supervisor, but it was then modified to provide more flexibility in frequency control.

A brief summary of the PLL configuration is as follows.
There are 3 reconfigurable stages from the input frequency $f_{in}$ to the output frequency $f_{out}$, called $M$, $N$, and $C$.
$M$ is a multiplier, and $N$, and $C$ are dividers, or in an equation,
$f_{out} = f_{in} \times \frac{M}{N \times C}$.
Each stage can be individually bypassed, and there can be multiple parallel $C$ stages in a single PLL to provide multiple frequencies outputs.

Converting from the desired divisor or multiplier to binary write data is not trivial, but it can be written into a function as all the rules are stated in the user manual~\cite{Altera7}.
After writing it in, the software waits for the hardware to finish configuring by watching another memory location, and then returns the frequency set.
This may deviate from the desired frequency due to hardware limitations of the PLL.

To illustrate the abstraction, we can examine what happens when the user writes the command \texttt{reset} to reset the entire system.

First the software calls the \texttt{cleanreset} function in class \texttt{wrapper}.
This calls the a function to set the reset bit and then clears it.
It also clears the enable signal and the freeze signal.
Then it writes all zeros to all driver filter control registers, completing the wrapper side reset.
Then the software calls a function in class \texttt{pll}, which sets the PLL frequency back to the default value, thus fully resets the whole system.

\section{Read-eval-print Loop}

To set up a REPL, a list of expressions that the user is allowed to give to the program is defined as in Table \ref{REPL Commands}.
These commands are made to be intuitive to the user, yet retaining all functional control of the system.

\begin{table}[H]
  \centering
  \begin{tabular}{|>{\ttfamily}p{11em}|p{\dimexpr\textwidth-18em}|}
    \hline
    \textrm{Command}   & Explanation \\
    \hline
    reset              & Resets the system and test results. \\
    version            & Prints the system version. \\
    freq <speed>       & Sets the clock speed to the specified value in MHz. Prints the actual frequency configured. \\
    mode <m|a>         & Choose between \underline{m}anual and \underline{a}uto test mode. \\
    manual <a|b> <hex> & Give input in manual mode. \\
    bitset <a|b> <hex> & Force bits to be 1 in auto mode. \\
    bitclr <a|b> <hex> & Force bits to be 0 in auto mode. \\
    run <time>         & Runs the test for specified duration in ms. Prints the results at the end of the test. \\
    exit               & Exits the REPL. \\
    \hline
  \end{tabular}
  \caption{Commands accepted in test REPL}
  \label{REPL Commands}
\end{table}

As user friendliness is a major concern in this project, the user inputs are not assumed to be always syntactically correct.
Therefore, the raw inputs are first sent through a series of checks to make sure they are valid.
This means the command must be in the list, they must be supplied with the correct number of arguments, and all arguments are in a valid form.
If any of the tests failed, the command will not ran and a helpful error message will be printed.
Otherwise, the command will be passed with a parser and translated to read and write instructions to the FPGA.

A debug feature is also provided in the software, so the user can choose to run the program in a sandbox first before running it on actual hardware.
This is done with the built-in constant \texttt{\_\_debug\_\_} in Python.
If the user decides to run the script in debug mode with \texttt{python ./run\_test.py}, the script will not actually write or read anything in the FPGA, but print out the address of all locations that it will be writing or reading in a real run.
To actually run the test, the user should use \texttt{python -O ./run\_test.py} to disable debug mode.

\section{Automation}

The initial disadvantage of a REPL program is that it requires inputs from the users every time they wishes to do something.
This can quickly get tedious, so to counter this issue, we have set up an automation system.
Now the users can write all the commands that they wants to run into a file, and the script will parse them the same as in REPL mode line-by-line.

This is implemented with a check for \texttt{argv} after the program starts.
If there is an argument provided when the program is called, as in \texttt{python -O ./run\_test.py test.do}, then the program will jump to the automated mode, and if there is no arguments, the program will start in REPL mode.
This syntax should be familiar to users, as this is how the \texttt{python} command and many other command line program works.
