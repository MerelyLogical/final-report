\chapter{System Implementation}

\section{Project Hierarchy}
Programming the FPGA to communicate with the HPS is no trivial task.
Luckily, there exists a golden system reference design~\cite{Rocket1} for the board in use for this project.
Unfortunately, support for certain versions of Quartus are missing from the GSRD download database, including the version used for this project, 16.0.
While the design can be opened with a different version of the software, it causes a series of conflicts usually related to using IP cores that have changed over the iterations.
To circumvent this issue cleanly, GSRD version 14.1 was downloaded and compiled on a separate install of Quartus II 14.1.
This allowed the reference design to be studied in detail, and the sections required for this project to be rebuilt with Quartus Prime 16.0.

\begin{figure}[H]
  \centering
  \begin{forest}
  for tree={
    align=center,
    font=\ttfamily,
    edge+={thick, -{Stealth[]}},
    l sep'+=10pt,
    fork sep'=10pt,
  },
  forked edges,
  if level=0{
    inner xsep=0pt,
    tikz={\draw [thick] (.children first) -- (.children last);}
  }{},
  [sys\_top
    [reset]
    [debounce]
    [soc\_system
      [hps]
      [hps\_only\_master]
      [jtag\_uart]
      [button\_pio]
      [...]
    ]
    [edge\_detector]
    [...]
  ]
\end{forest}
  \caption{Hierarchy of the Golden Reference Design}
  \label{Golden Hierarchy}
\end{figure}

By examining Figure \ref{Golden Hierarchy}, the structure of the reference design, we see that it has a top-level wrapper called \texttt{sys\_top}, which instantiates the Qsys system \texttt{soc\_system} and a few IP blocks that handle the low-level hardware controls on the development board.
This Qsys system is of the most interest to us as it contains the module \texttt{hps}.
In \texttt{hps}, there are 3 ports named \texttt{h2f\_axi\_master}, \texttt{h2f\_axi\_slave}, and \texttt{h2f\_lw\_axi\_master}, corresponding to the bridges exposed by the HPS for connections~\cite{Altera6}.

With this knowledge, we can insert the testbench design into this hierarchy by having it wrapped into a Qsys module, with an open port that works as a slave to the AXI bridge.
As the traffic passing through the HPS-FPGA bridge is minimal in our design, the lightweight bridge will be used for its simplicity.

As the testbench only requires a list of registers to be sparingly read from and written to, the logic required for the signals on the Avalon slave interface can be handwritten in the module \texttt{text\_wrapper} according to the interface specifications~\cite{Intel3} without much trouble.

By following the naming conventions, the signals allow Qsys Component Editor to automatically detect the Avalon slave from this module during analysis.
This saves the trouble of editing the \texttt{\_hw.tcl} file.

Since Qsys was chosen as the user interface during design, it makes sense to also put the DUT and the reference design at this level in the hierarchy.
This means that the user only needs to use Qsys to swap in new designs.
As such, other than handling the interface to the HPS, the wrapper module also exposes conduits that connect to the DUT and the references designs.

In addition, the PLL and its reconfiguration IP cores are also instantiated at this level.
This is convenient as Qsys is designed for integrating such IP cores.

\begin{figure}[H]
  \centering
  \begin{forest}
  for tree={
    align=center,
    font=\ttfamily,
    edge+={thick},
    l sep'+=10pt,
    fork sep'=10pt,
  },
  forked edges,
  if level=0{
    inner xsep=0pt,
    tikz={\draw [thick] (.children first) -- (.children last);}
  }{},
  [\textbf{sys\_top}
    [reset]
    [debounce]
    [\textbf{soc\_system}
      [\textbf{hps}]
      [\textbf{pll}]
      [\textbf{pll\_reconfig}]
      [\textbf{test\_wrapper}
        [\textbf{testbench}
          [randomiser]
          [driver]
          [monitor
            [sub\_mon]
          ]
          [scoreboard]
        ]
      ]
      [\textbf{dut}]
      [\textbf{ref}]
      [...]
    ]
    [edge\_detector]
    [...]
  ]
\end{forest}
  \caption{Hierarchy of the Full Hardware System}
  \label{Hierarchy}
\end{figure}

The main section of the testbench is instantiated in the wrapper as a separate module, which is named \texttt{testbench}.
This module instantiates and connects all main components of the testbench.
While this module seems unnecessary, being another wrapper in a bigger wrapper, it does provide 2 advantages.

Firstly, it simplifies the development cycle of the framework.
During the compilation of a Qsys system, all relevant files are copied into a folder and the system is transformed into a Verilog file that has its direct dependencies contained in that folder.
While this is arguably a benefit in terms of dependency management, it makes the development more difficult, since whenever something is changed in the interface between testbench modules, the entire \texttt{soc\_system} needs to be recompiled to update the dependencies.
If forgotten, this can cause confusion as the testbench could still be the old version even though a new compilation at the \texttt{sys\_top} level has been performed.

Secondly, it makes the simulation of the testbench more straightforward.
Verifying the correctness of the Avalon slave in the wrapper is important, but there is no need to go through the interface protocol whenever a new input signal is desired in testbench simulations.
Having the \texttt{testbench} module allows direct manipulation and examination of the signals in and out of the testbench, without worrying about the HPS in simulation.
