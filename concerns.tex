\section{Ethical, Legal, and Safety Plan}
% The Ethical, Legal and Safety Plan must detail what are the issues in this are
% relevant to your project, showing how you will comply with best practice. If
% there are no such issues (the case for 80\% of all projects) you must
% nevertheless show here that you have considered these issues and detail why they
% will not apply to your project. Information will be provided on the project web
% pages about Ethical, Legal and Safety matters.

\subsection{Ethical Considerations}
% Ethics: Research projects are required to get ethical approval if they may have
% issues. It is not at all usual for EEE FYPs to have such issues, but you should
% be aware of the possibility, most usually relating to personal data when you
% conduct questionnaires or do user testing.

Checking against the ethical issue list provided by Imperial College Research
Ethics Committee~\cite{Imperial1}, this project
\begin{itemize}
  \item does not damage participants' mental or physical health;
  \item does not jeopardise the safety and liberty of the researchers;
  \item does not use any private information;
  \item does not involve sensitive subject matter or methods;
  \item does not risk any conflict of interest between the researchers and
        the College.
\end{itemize}
This project is thus free from significant ethical concerns.

\subsection{Legal Considerations}
% Legal issues: the obvious one here is patent protection. Your work may infringe
% patents (which does not affect the project itself, but might affect
% commercialisation). You are not required to do a patent search but should be
% aware of the possibility if you have a deliverable that could be commercialised.
% Also, your work may (if it goes well) be patentable. You can usually check this
% with your supervisor. If so you will need to ensure that nothing you do
% prejudices this possibility. There may also be regulatory issues with
% electromagnetic radiation.
% REVISIT: ask James about legal issues
Intel Quartus Prime software offers a variety of IP cores.
These are encrypted module designs that would be integrated into the
verification system of the project~\cite{Intel2}.
Intel FPGA Evaluation Mode allows

\subsection{Safety Considerations}
% Safety: check with your supervior (who is ultimately responsible for safety in
% your lab work). Also, you can get information from the lab staff - e.g. Vic
% Boddy - about possible safety issues in the lab. Note that Vic is not
% responsible for safety, though quite knowledgable, so you must also check this
% with your supervisor. Many projects will have no lab work and therefore no
% safety issues.
As the project is done mainly on a computer with minimum physical aspects,
there is no major safety concern.
For the minor concerns associated with the project, 
the physical development board will be handled with care, and the desk works
will be interleaved with breaks.