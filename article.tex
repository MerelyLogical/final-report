\documentclass[12pt]{article}
\usepackage[margin=2cm]{geometry}
\usepackage{cite}
\usepackage{float}
\usepackage{graphicx}
\usepackage[caption=false]{subfig}
  \DeclareGraphicsExtensions{.png}
\usepackage{amsmath}
\usepackage{amsfonts}
\usepackage{url}
\usepackage{tikz}
\usetikzlibrary{shapes,arrows}
\usepackage{tikz-timing}
\usetikztiminglibrary[new={char=Q,reset char=R}]{counters}
\usepackage{bm,times}
\usepackage{pgfgantt}
\usepackage{indentfirst}
\usepackage{array}

\begin{document}
\begin{titlepage}
  { \Large
    Imperial College London\\[17pt]
    Department of Electrical and Electronic Engineering\\[17pt]
    Final Year Project Report \textcolor{red}{[DRAFT]}
  }

  \rule{\columnwidth}{3pt}
  \vfill
  \centering
  \includegraphics[width=0.7\columnwidth]{img/1.jpg}
  \vfill

  \begin{table}[h]
  \def\arraystretch{1.8}
    \begin{tabular}{p{40mm}p{\dimexpr\columnwidth-40mm}}
      Project Title: & \textbf{An Extensible Framework for \newline At-speed Evaluation of Arithmetic Hardware} \\
      Student:       & \textbf{Zifan Wang} \\
      CID:           & \textbf{01077639} \\
      Course:        & \textbf{EEE4} \\
      Project Supervisor: & \textbf{Dr. James J. Davis} \\
      Second Marker: & \textbf{Dr. Christos Bouganis}
    \end{tabular}
  \end{table}
\end{titlepage}

% \markboth{2018-2019}{?}
\setcounter{tocdepth}{2}
\tableofcontents

\newpage

\begin{abstract}
  Nice abstract
\end{abstract}

\section{Introduction}

\section{Background}

\section{Requirements Capture}

\section{Analysis and Design}

\section{Implementation}

\subsection{Randomiser}
\subsubsection{Fibonacci vs Galois}
\subsubsection{Vertical vs Horizontal}

\subsection{Driver}
\subsubsection{Dual Driver System}
\subsubsection{Delay Tester}

I built a delay tester to find out the delay of the DUT.
With a 3-bit counter as shown in the timing diagram, it can measure this delay for up to 8 clock cycles.

\begin{figure}[ht]
  \centering
  \begin{tikztimingtable}
  [
    xscale=2.5,
    timing/d/background/.style={fill=white},
    timing/font=\ttfamily
  ]
  out\_count       & 6R 2{Q} 0R 8{Q} 0R 8{Q} 0R Q\\
                   & L 2{H 7L}       HL          \\
  o\_drive         & U 2{D{0} 7U}    D{0}U       \\
  i\_dut\_out      & U 2{3U D{0} 4U} 2U          \\
  test\_state      & 5D{00} 5D{01} 3D{10} 6D{11} \\
  delay\_out       & 11D{0} D{1} D{2} 6D{3}      \\
\extracode
  % Add vertical lines in two colors
  \begin{pgfonlayer}{background}
    \begin{scope}[semitransparent,semithick]
      \vertlines{1,2,...,18}
    \end{scope}
  \end{pgfonlayer}
\end{tikztimingtable}
  \caption{3-bit Delay Tester FSM}
  \label{DelayTester}
\end{figure}
Testing with 0 is safe since LSFR will never output 0.

\subsection{Monitor}
\subsubsection{Sub Monitors}

\begin{figure}[ht]
  \centering
  \begin{tikztimingtable}
    [
      xscale=4,
      timing/d/background/.style={fill=white},
      timing/font=\ttfamily
    ]
    dist\_ctr     & D{8} 2{D{1}D{2}D{4}D{8}} D{1}D{2}      \\
    a\_dut & U D{AA}D{AB}D{AC}D{AD}D{AE}D{AF}D{AG}D{AH}D{AI} U \\
    b\_dut & U D{BA}D{BB}D{BC}D{BD}D{BE}D{BF}D{BG}D{BH}D{BI} U \\
    s\_dut & 3U D{SA}D{SB}D{SC}D{SD}D{[red]SK}D{SF}D{SG}D \\
    dist\_ctr [0]  & L    2{H 3L}             HL \\
    a\_mon         & U 4D{AA} 4D{AE} 2D{IA} \\
    b\_mon         & U 4D{BA} 4D{BE} 2D{IB} \\
    s\_mon         & 2U 4D{SA} 4D{SE} D \\
    sub\_event [0] & 8LH2L \\
  \extracode
    % Add vertical lines in two colors
    \begin{pgfonlayer}{background}
      \begin{scope}[semitransparent,semithick]
        \vertlines{1,2,...,10}
      \end{scope}
    \end{pgfonlayer}
  \end{tikztimingtable}
  \caption{Distributed Monitoring System}
  \label{DisMon}
\end{figure}

\subsection{Scoreboard}

\section{Testing}

\section{Results}

\section{Evaluation}

\section{Conclusion}

\section{Further Work}

\section{User Guide}

\newpage
\appendix

% \chapter{User Guide Provided to Test Volunteers}

\chapter{Raw Data}

\textcolor{red}{+Code Snippets?}


\input{interim/bibliography}

\end{document}