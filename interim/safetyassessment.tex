\documentclass[11pt]{article}
\usepackage[margin=2cm]{geometry}
\usepackage{cite}
\usepackage{float}
\usepackage{graphicx}
\usepackage[caption=false]{subfig}
  \DeclareGraphicsExtensions{.png}
\usepackage{amsmath}
\usepackage{amsfonts}
\usepackage{url}
\usepackage{tikz}
\usetikzlibrary{shapes,arrows}
\usepackage{bm,times}
\usepackage{pgfgantt}
\usepackage{indentfirst}
\begin{document}

\title{%
  A High-radix Online Arithmetic Verification System\\
  \large Final Year Project 1800478: Safety Assessment}
\author{Zifan Wang, 01077639\\Imperial College London}

% \markboth{2018-2019}{?}
\maketitle

\textbf{Electrical safety}\\
There is no high voltage present on the development board.
\\

\textbf{Physical safety}\\
There are no large or fast-moving objects involved.
\\

\textbf{Chemical safety}\\
There is no poisonous or irritant or allergenic material.
\\

\textbf{Fire safety}\\
It is possible that the SoC can get overheated.\\
This should not result in a fire as the SoC is actively cooled and the system
is temperature aware.
\\

\textbf{Biological safety}\\
There are no biological hazards.
\\

\textbf{Animal safety}\\
There are no animals involved.
\\

\textbf{Appliance safety}\\
The board is certified by the department.\\
The development tools can detect certain potentially damaging
configurations and estimate power consumption.\\
I will also care to avoid such designs\footnote{A. Agne et al, ``Seven recipes
for setting your FPGA on fire -- A cookbook on heat generators''.} when
constructing the project.
\\

\textbf{Airspace safety}\\
There is no use of airspace.
\\

\textbf{Study Participant safety}\\
There are no additional participants.
\\

\textbf{Data Infrastructure safety}\\
It is useful to be able to access the development board remotely.\\
I will ensure to be doing so in an safe environment while using secure methods
of connection.

\end{document}