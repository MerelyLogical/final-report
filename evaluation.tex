\section{Evaluation Plan}
% The evaluation plan should detail how you expect to measure the success of the
% project. In particular it should document any tests that are required to ensure
% that the project deliverable(s) function correctly, together with (where
% appropriate) details of experiments required to evaluate the work with respect
% to other products or research results.

\subsection{Metrics}
One natural way of measuring the success of the project is to look at the actual
progress and comparing it to the plan given in the implementation plan.
It should be noted that no plan is perfect, so some deviation is allowed.
However, if there is significant delay from the implementation plan, there must
be justifications given.

The next few measures looks at the performance of the final product.
First, the maximum stress of that the testbench and provide without failing can
be used as a success measure.
A testbench with a higher maximum frequency can reveal a wider picture in the
performance of the DUT.
This would hopefully allow more insights to be gained regarding the DUT, or
it could mean that the testbench can be used for future designs that may be
faster than the current one.
As the main quantitative metric, this would be a vital indicator of the
project's success.

The Robustness of the testbench is also vital to the product's performance.
The testbench should be free from errors within a reasonable operating range.
If the testbench becomes unreliable with some minor changes to the system,
the data that can be obtained would be very limited.
The failure of the DUT can no longer be confirmed, as the error may be in the
testbench instead of the DUT.

The ease of use of the testbench could be another evaluation point.
On the hardware side, the verification system can be packaged into a Qsys
module.
Given the DUT is also a module with an agreed interface, they could be easily
connected in Qsys for testing.
For example, the DUT may be written in VHDL for its deterministic nature, but
the testbench maybe written in Verilog for its simplicity, but as both can be
synthesised into a module, they would still be compatible in Qsys.
On the software side, a user-friendly interface could be built.
A usable command line interface maybe good enough, but a simple graphic
interface could make the tests much more visual and interesting.

The interface could also provide information on the failure in the DUT.
A better testbench would provide more insightful details when the DUT fails.
This would make debugging or evaluating the design much simpler.
Along with the GUI, this project has many optional extensions that would be
discussed further in the corresponding section.
After the main goal of the project being met, the number of optional functions
implemented would become a good measure of the progress of the project.

Since the project is of the verification system, the results from the benchmarks
should not be used for evaluation of this project.

\subsection{Risks and Fallbacks}
A major risk of this project is related to its schedule and the existence of an
initial blocking task.
While most of the later sections of the project can be selectively added or
removed from the scope relatively easily, the initial setup of the testbench
structure will always remain critical to any further improvements.
It is thus vital that the bare minimum system gets done early.
To ensure this happens, it will be placed in the highest priority before its
completion, and any blocking issue should be discussed with the supervisor if it
could not be resolved after significant effort.

The other major risk has to do with the progress of the sister project.
The purpose of the testbench is to verify and stress the arithmetic designs.
If these designs would not be available near the end of this project,
it would be difficult to empirically prove the capabilities of the testbench
and its surrounding system.
It is not impossible, as there are still substitutions for them.
For functional purposes, standard off-the-shelf arithmetic modules could be
used in-lieu.
For other purposes, it is possible to have a model done before the actual design
starts in the paired project.
While this would allow this project to progress easier, it would be extra work
for the other project, which is ultimately up to the decision of the other
student.
In all, it would be nice to have a solid arithmetic module completely to run
in this testbench, but without one, the system can still be built and completed,
albeit generating less useful data towards the overall aim of the project.

\subsection{Extensions}
