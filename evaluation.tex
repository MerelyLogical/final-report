\section{Evaluation Plan}
% The evaluation plan should detail how you expect to measure the success of the
% project. In particular it should document any tests that are required to ensure
% that the project deliverable(s) function correctly, together with (where
% appropriate) details of experiments required to evaluate the work with respect
% to other products or research results.

\subsection{Metrics}
In order to evaluate the progress and success of the project, the difficulties
of the deliverables need to be analysed first.



\subsection{Risks}
A major risk of this project is related to its schedule and the existence of an
initial blocking task.
While most of the later sections of the project can be selectively added or
removed from the scope relatively easily, the initial setup of the testbench
will always remain critical to any further improvements.
It is thus vital that the bare minimum system gets done early.
To ensure this happens, it will be placed in the highest priority before its
completion, and any blocking issue should be discussed with the supervisor if it
could not be resolved after significant effort.

The other major risk has to do with the progress of the sister project.
The purpose of the testbench is to verify and stress the arithmetic designs.
If these designs would not be available near the end of this project,
it would be difficult to empirically prove the capabilities of the testbench
and its surrounding system.
It is not impossible, as there are still substitutions for them.
For functional purposes, standard off-the-shelf adders and multipliers could be
used in-lieu.
For other purposes, it is possible to have a model done before the actual design
starts in the paired project.
While this would allow this project to progress easier, it would be extra work
for the other project, which is ultimately up to the decision of the other
student.
In all, it would be nice to have a solid arithmetic module completely to run
in this testbench, but without one, the system can still be built and completed,
albeit generating less useful data towards the overall aim of the project.